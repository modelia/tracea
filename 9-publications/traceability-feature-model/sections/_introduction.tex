\section{Introduction} \label{sec:intro}

The need for traceability has always been salient in software and systems development. Across the years, there has been a continuous interest in developing techniques to facilitate the representation and analysis of traces and links between related artefacts. It helps explaining their execution and evolution as required  in many software engineering activities and disciplines such as code-generation, program understanding, software maintenance, and debugging. 

The importance of traceability was first recognized in system engineering especially related to the development and certification of critical systems where it is a primary concern. As an example, traceability is part of any certification mechanism in all commercial software-based aerospace systems as stated in documents like the RTCA DO-178C (2012)~\cite{paz2019-Modelling-Avionics-Certification,moy2013-DO-178C-testing}. The consideration of various levels of abstraction in software development and the meaning of verification in model-based development paradigm - which figures abstract representations (models) as the core artefact for conceptualization -  was latter introduced with companion documents (specifically, DO-331). The automotive industry has followed the same path with the construction of an international standard for functional safety~\cite{iso26262}. %At code, requirement, model, human and organisational level, a cross-sectorial view provided through traces of processes' executions and artefacts' dependencies figures an ideal tool support for software development and maintenance.

Despite these important evidences on the need for explicit (and automated) tracing abilities in software development, traceability is not widely adopted, even less automated, and there is little feedback from its concrete use in industry~\cite{panis2010-req-traceability-deployment-in-commercial-engineering-organisation} beyond the critical domains above. 

There is a lack of global techniques to ease the manipulation of traces and automate tracing processes. Thereby, traceability in the industry, when required, ends up being mostly a manual process~\cite{mader2009-motivation-matters-in-traceability-practitioner-survey}.
%Winkler \textit{et al.} show all the disregard such "mandatory" characteristic implies - expressly to safety critical software demanding a legal assessment or certification to run~\cite{winkler2010-survey-traceability-and-MDE}. Mader \textit{et al.} show that the benefits of automated traceability is still to be justified to stakeholders who still prefer the burden of \textit{ad hoc} and manual trace elicitation~\cite{mader2009-motivation-matters-in-traceability-practitioner-survey}. 
Moreover, with no standard definition or representation of traces, it is difficult to bridge the gaps between the different partial traceability solutions existing in research subfields\cite{antoniol2017-traceability-grand-challenges,wohlrab2020-traceability-organization-process-culture,winkler2010-survey-traceability-and-MDE}.   %a gap remains between broader fields of investigation beneficial from automated tracing ability (\textit{e.g.,} requirement engineering, program understanding, testing). As a matter of fact, 
Even the software engineering body of knowledge do not seem to properly consider the key relevance of traceability in software engineering as it only mentions traceability once~\cite{swebok2014}.
%Many studies suggest that there is a lack of synthetic and systematic investigations on traceability and research is scattered among many sub-fields. This lack of a common tool for communication hampers sharing between research teams~

All this in a context where artificial intelligence techniques are being integrated in development processes, raising the need for more powerful reproducibility and explainability concerns, both requiring the assistance of traceability mechanisms. %and therefore explainability needs are increasing. Yet, there exists no silver bullet to solve the different expectations brought by tracing ability. Approaches vary greatly in their means and goals. The foundation for an effective modelling of traceability is disseminated among a profuse literature. Moreover, most approaches focus on specific pairs of artefacts and therefore remain difficult to integrate in industrial scenarios.

%the paper aims to help readers understand all the dimensions they should keep in mind when evaluating traceability approaches.
This paper aims to provide a comprehensive perspective on the state of the art of traceability techniques in software development and their limitations with the short-term goal of facilitating the evaluation and comparison of current solutions. And with the mid-term goal of accelerating the development of new traceability solutions that could benefit from the existing ones thanks to our new conceptualization in the form of a feature model describing the potential dimensions and concerns a traceability solution may wish to consider.   %We ground our work with a survey that target publications involved in the modelling of traceability to reflect on the emergence of the field.
%This is the motivation to classify papers according to a series of dimensions that should facilitate the creation (or combination) of more powerful traceability solutions in the future. To help compare current and future traceability proposals we provide a feature model that aims to organize and structure the main traceability concepts. 
We do not create the feature model or just based on our (partial) knowledge and expertise in the domain. Instead, we ground our classification with a survey of the published literature in this field. According to this survey, we group the traceability features in three main dimensions: trace definition, trace identification and trace management, with the corresponding feature hierarchies for each of them. 

%[Metamodeling needed for communication~\cite{winkler2010-survey-traceability-and-MDE}]
%[Metmodeling ISO/IEC 24744:2007 does not mention traceability.\ugh{Check 24744:2014}]

The paper is organized as follows. After a brief introduction, we discuss in \Sect{sec:soa} how our work compares to other meta studies and characterizations of traceability research. We then introduce some basic traceability terminology in \Sect{sec:terminology}. Section~\ref{sec:survey} describes how we conducted our literature review and \Sect{sec:fm} presents a detailed feature model derived from the survey of the retrieved works. This analysis also helps us to propose a number of discussion points and open challenges in \Sect{sec:discussion} before concluding this work. 


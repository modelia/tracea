\section{Discussion}\label{sec:discussion}

The feature model is a first step towards the shared understanding of all dimensions involved in a traceability solution. Ideally, a company interested in a certain set of such dimensions could try to create its perfect traceability solution by combining the top solutions for each dimension. But this is not yet a real possibility as those solution would be difficult to combine and, more importantly, several of the features in the feature model do not really have a great solution yet. This section elaborates on this discussion by presenting some open challenges in software traceability research.

\textbf{Common traceability metamodel}. We have counted over 20 different metamodel proposals. Some are solutions to specific problems the authors present as case studies. And these metamodels are rarely reused, if ever. This proliferation is a challenge to make different traceability solutions interoperate. The research community should agree in a unified proposal that facilitates the composability of traceability solutions. We believe Eclipse Capra~\cite{heisig2019-generic-traceability-metamodel-end-to-end-capra}, even though build to address software product line tracing, could provide a solid foundation for this unified metamodel as it already comes with good tool support to build on.\ugh{Precise: for MBSE.} \ugh{Customization and visualization tooling. Generic link definition. External language based on Ecore/XMI.}

\textbf{Complete traceability metamodel}. Following up on the previous point, to agree on a core traceability representation may not seem difficult but it would ignore many of the aspects in the feature model that we believe are key in any non-trivial and industrial traceability application, such as the quality and temporal annotation of traces. A core model with an extension mechanism could be a good compromise here.

\textbf{Security of trace data}. Considering that traceability is a major aspect in certification and other critical applications, it is surprising to see very little interest in security concerns related to trace artefacts. We believe security mechanisms (even simple rule-based access control) for traceability are needed to control who can modify what trace data, given the implication such changes can have. 

\textbf{Library of trace types and semantics}. We already mentioned the importance of having a rich set of types for traces to let engineers express the reasons behind the creation of a given trace. But at the same time, complete freedom makes reusability of analysis techniques difficult. We would like to see a rich yet predefined set of types for traces that could then be imported in new traceability projects.

\textbf{Usefulness of identified traces}. Managing a large number of traces is time consuming. As such, we should make sure every explicit trace is actually useful. So far, algorithms aimed at automatically identifying traces are compared based on standard properties like precision and recall. But they should be evaluated on ``usefulness'': are those traces useful for the end-user? or are just redundant noise? 
%hese questions have been long due and a common conceptualization of the core tracing abilities will be a first step to distinguish between tracing purposes.
%The efficiency of identification is primarily based on the measurement of precision and recall of the algorithms. This is a widely accepted limitation since this quantification does not take into account what the actually identified traces represent for the end user – are they “useful”? And how? These questions have been long due and a common conceptualization of the core tracing abilities will be a first step in that direction. 

\textbf{Verification, validation and testing of traces}. Our ample literature on verification, validation and testing methods for software engineering should be extended to deal with trace data, especially from a temporal perspective, where temporality would depend on pure timestamp values (i.e. how long since the trace was created) and on evolution lag (i.e. how many times the linked artefacts have changed since the trace was created). Reasoning on outdated and potentially incorrect trace data could have strong damaging impacts on the system as a whole. So far, very few approaches target these aspects except for the specific  problem of coevolution in model-driven engineering. 
The ability to justify – with evidences and uncertainty evaluation – the quality and integrity of traces is a prerequisite to robust and reliable traceability. And given the effort required to create traces in the first place, this is important to instill more confidence to practitioners wondering whether creating traces is worthwhile.

\textbf{Traceability as first-class concern in general languages}. Another important step towards the mainstream adoption of traceability in industry is the integration of the common traceability metamodel in popular modeling languages like UML or SysML, in the form of a profile (to be able to directly reuse existing modeling tools available for those languages) or new packages in the respective standards. This way, traceability would become a first-class citizen in software development while still being a rich concept and not just the plain dependency relationship we can use right now in those languages. 

\textbf{Working together with Industry}. Orthogonal to all the others, we (the research community) should aim to have more frequent exchanges with practitioners to better understand why they end up creating traces manually instead of trying to reuse any of the dozens existing solutions covered in our survey. Some reasons have been already hinted in this paper, based on our own experience in industrial projects involving some type of traceability need and based on the survey we have conducted,  but there could be others we are not aware of. Or a different prioritization than the one we have in mind. If we want traceability research to transfer to industry, more and better communication flows should be part of the agenda. 


%Parallel to these facts, we see a strong dichotomy between generalist and specialist investigations. On the one hand, researchers attempt to build generic approaches and tools and gather material for the pooling of knowledge. Heisig \textit{et al.} offer a fit-for-all metamodel based on (and restricted to) Ecore technology~\cite{heisig2019-generic-traceability-metamodel-end-to-end-capra} ; Tekinerdogan \textit{et al.} present a metamodel for tracing across the software life cycle~\cite{tekinerdogan2007-metamodel-for-tracing-concers-accross-life-cycle} ; Keenan \textit{et al.} present a workbench to evaluate traceability approaches~\cite{keenan2012-workbench-for-traceability}). The adoption of these approaches is hampered by inherent design choice (\textit{e.g.,} using Ecore, putting software life cycle as the main factor)

%The most common objective seems to be the identification of traces. The use of AI has helped a lot develop this important feature for end-to-end traceability. Adaptations and optimizations of machine learning techniques to the identification of traces are numerous. Unfortunately, their adoption by industry remains little. This is due in part to the methods adopted by researcher to validate their work. 
% which remain mainly an arbitrary (precision and recall) quantification. 
%Where the shoe pinches is in the ability to evaluate the benefits of increasing the tracing ability of software outside of comparison between techniques. 


%In this regard, the importance of the typing of links and artefacts is unanimous. It is about allowing the user to manipulate the relational elements and their targets in the language of his or her expertise. The granularity of the target artefacts must be adapted to the field of application as well~\cite{clelandhuang2007bestPracticeForAutomatedTraceability}. 
%Studies aiming at powerful generalization offer mechanisms for customizing relationships and their targets. 


%\ugh{[Industry is rather incline to (re)do manually the tracing instead of automating something that will be obsolete in less time than it takes to tell.]}
 
%Our survey show a proliferation of metamodels.  . These specific applications of {conceptual explorations} has shown the importance of a dedicated language external to the environment of the system. Independent representations increase the robustness and the adaptability of tracing solutions to heterogeneous systems~\cite{mustafa2017-literature-review,Dubois_2010}.

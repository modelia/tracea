\begin{abstract}
Traceability is the capability to represent, understand and analyze the relationships between software artefacts. Traceability is at the core of many software engineering activities. This is a blessing in disguise as traceability research is scattered among various research subfields which impairs a global view and integration of the different innovations around the recording, identification and management of traces. This also limits the adoption of traceability solutions in industry.

In this sense, the goal of this paper is to present a characterization of the traceability mechanism as a feature model depicting the shared and variable elements in any traceability proposal. The features in the  model are derived from a survey of papers related to traceability published in the literature. We believe this feature model is useful to assess and compare different proposals and provide a common terminology and background that could speed up the creation of new ones on top of them. Beyond the feature model, the survey we conducted also help us to identify a number of challenges to be solved in order to move traceability forward, especially in a context where, due to the increasing importance of AI techniques in Software Engineering, traces are more important than ever in order to be able to reproduce and explain AI decisions.


%Developing techniques to facilitate the representation and analysis of software and system artefacts has always been under the scrutiny of researchers. More particularly, tracing links between related artefacts has shown significant value. Traceability helps explaining the execution and evolution of software systems and assists the progress of primary software engineering disciplines such as program understanding, maintenance, and debugging.
%Yet, the results of traceability investigations are scattered among various research fields and a common ground for theorizing and conceptualizing traceability remains slow to emerge. This is, at least in part, due to a lack of interest from stakeholders as well as remaining of the idea that traceability is of no use to software development efficiency and quality.
%In this paper, we argue that the many purposes traces enable to software engineers share a strong common base. We gather them in a feature model that aims to help readers understand all the dimensions they should keep in mind when evaluating traceability approaches. We ground our approach on a methodological selection of publications referring to the modeling of tracing abilities as well as to tracing itself.% to derive commonalities among features and dimensions.

\end{abstract}

% Note that keywords are not normally used for peerreview papers.
%\begin{IEEEkeywords}
%Computer Society, IEEEtran, journal, \LaTeX, paper, template.
%\end{IEEEkeywords}}


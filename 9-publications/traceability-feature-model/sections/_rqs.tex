\section{Research questions}\label{sec:rqs}

Beyond the overall goal of analyzing the state of the art in traceability research in software/systems engineering and identifying its open research challenges, we are especially interested in the following subtopics that will be the main focus of our analysis when evaluating the papers resulting from the survey.

\subsection{RQ1: How are traces represented?}
\label{sec:rq1}
We are interested in knowing what languages are used to represent traces and how expressive they are. E.g., is it possible to define types of traces? And use them to link clusters of artefacts (or just individual elements)? Can we express the confidence we have in those traces?

\subsection{RQ2: How are traces identified?}
\label{sec:rq2}
Do traces need to be manually created or can they be derived from existing information? If so, what methods can be used for such derivation? These are some of the topics we would like to answer in this RQ.

\subsection{RQ3: What tool support is available to manage and apply traces?}
\label{sec:rq3}
Finally we want to draw a general map of the techniques that traceability approaches use to store and maintain trace artefacts of quality.

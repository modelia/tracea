\section{Conclusion}\label{sec:conclusion}
Traceability research is scattered among different software engineering subfields resulting in diverse but partial solutions to represent traceability information. We have presented a complete traceability metamodel that aims to cover all its aspects, including the quality and uncertainty of specific traces, their decay or the evidences that support them. This information is needed to make fully informed decisions based on trace data. Our proposal has also been designed with modularity and extensibility principles in mind to facilitate its adoption in a large variety of domains. We believe it should help in improving a number of traceability-based algorithms (e.g. for impact change analysis) that could now also take into account these additional traceability dimensions.

As further work we want to continue advancing on these latter aspects, mainly proposing extensions to general modeling languages (like SysML or UML) that integrate our traceability metamodel. Moreover, we will explore the complementarity of AI and traceability. Regarding AI for traceability detection we plan to extend existing techniques to automatically infer traces to populate our metamodel considering the integrity and quality aspects of the inference process. Regarding traceability for AI, we plan to rely on our metamodel to offer better explainability support to the myriad of AI-based solutions for Software Engineering that right now mostly ignore this aspect~\cite{carleton2020-intersection-AI-and-SE,ozkaya2020-differences-in-AI-enabled-engineering}.
%We will study the application of this metamodel to AI applications in software engineering and its integration in general purpose modeling languages, like SysML, as hinted above.
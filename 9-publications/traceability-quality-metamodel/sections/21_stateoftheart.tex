	\vspace{-0.4truecm}
\section{State of the art}\label{sec:soa}
	\vspace{-0.2truecm}

We have identified over 80 approaches aimed a modeling traces and tracing activities~\cite{batot2020-survey-driven-feature-model}. We describe in this section a selection of the most representative publications and then summarize their main limitations as the key challenges our solution will aim to overcome. 

%These contributions span over three main areas:
%i)   approaches addressing requirement traceability \ugh{(Historically prominent)};
%ii)  approaches modelling traces generated by the engineering of automated process \ugh{(Modelling traces)};
%iii) approaches that aim at a generic solution for traceability \ugh{(Modelling traceability)}. In this section, we present these contributions and summarize thier principal limitations.

Most works on modelling traceability come, historically, from the requirement modelling community. Traces are seen as "links" from requirements to their (sub)components, and to their design and/or implementation artefacts \cite{taromirad2012-TIM-for-Agile-req,Haidrar_2018}. 
Specially relevant is the work by Goknil \textit{et al.} \cite{goknil2008-metamodel-for-reasoning} that includes a metamodel for traces, mechanisms for consistency checking and inferencing, and tooling for change impact analysis.

In the model-driven engineering (MDE) community, research work can be classified into proposals focusing on modeling the heterogeneity of artefacts -- with numerous contributions aiming at linking text artefacts to design models \cite{sannier2012-TIM-for-text-req-in-MDE}, or addressing the entanglement of (non functional) requirements \cite{yrjonen2010-TIM-for-NonFunc-in-MDE} -- and proposals adapting traceability to specific application areas such as the automotive and robotic industry  \cite{dubois2010-TIM-for-Req-in-MD-Automotive,Sanchez_2011,seibel2012-efficient-traceability-for-MDE}. In this group, we see several publications that include \textit{custom} metamodels built \textit{ad hoc} to solve specific model transformation issues \cite{levendovsky2010-TIM-for-MT,Jim_nez_2013}. %These cases show a specific target that forces modelling to fix artefacts and links types.
%Seibel \textit{et al.} model MDE tasks with precision and leave latitude to the user regarding the definition of relationships . 

Other modeling approaches are aimed at establishing an automated trace generation process, e.g. for requirement traceability. For example, Spanoudakis \textit{et al.} 
%\cite{spanoudakis2004-rule-based-generation-of-req-traceability-relations})
We see these cases as sidesteps from traceability modeling since the works aim at generating traces (that need being modelled) rather than modelling traces (that need being generated). As a consequences, the presented metamodels are specific to the types models source or target of the generation  (\textit{e.g.,} BPMN models \cite{pavalkis2017-TIM-for-BPMN}, or data warehouse models \cite{mate2011-TIM-for-MDA-data-warehouses}).
Natural language is also often used in this type of automated processes. In this case, approaches  target the extraction of semantics (or \textit{meaning)} from textual requirements. These publications model text blocks with their dependencies and the dependencies to specific third party artefacts (\textit{e.g.,} for MDE: \cite{sannier2012-TIM-for-text-req-in-MDE}, for AADL \cite{wang2020-TIM-for-NL-to-AADL}, for agile user stories~\cite{carniel2018-TIM-for-Impact-analysis-agile}). 

%These are important work that made researchers formalize the main functional requirements for tracing software product: offering the user fine grained artefacts and relationships as raw material tailored to their purposes. 

Recently, researchers attempted more generic approaches to traceability, closer to our own goal. Building on previous knowledge in specific domains, authors describe their attempts to synthesise traceability requirements. For example, Azavedo et al. \cite{azevedo2019-traceability-metamodel-and-reference-model}  created a metamodel with explicit (57) relationship types and (12) different kinds of artefacts based on an arbitrary separation of software development tasks (\textit{e.g.,} Implementation, Verification, Modification, Homologation). 
On the other hand, Heisig \textit{et al.} present a modeling approach to traceability that includes both a basic metamodel with a plugin mechanism (using XText) that allows user to define their specific representations for links and artefacts \cite{heisig2019-generic-traceability-metamodel-end-to-end-capra}.

While these latter approaches do represent in advance in the generalibility and adaptability of traceability metamodels, our approach offers a higher granularity and decomposition while integrating several quality concerns (decay, confidence, and explainability). Table \ref{table:occurence-of-qualities} summarizes existing works regarding these core traceability aspects. As shown in the table, most publications consider a single trace level, which limits the complexity and diversity of problems where traceability can be applied. There is also a significant lack of consideration for quality aspects. Consistency is merely mentioned and confidence is strictly forgotten -- none of the selected approaches mention it. Explainability is reported in a few cases but remains scarce and no common appreciation has emerged yet. 

%Traceability is used for specific cases and its product is thrown away after use. 
%We hypothesis benefits to traceability if traces are appreciated with deeper scrutiny. To this extend we define four main qualities: \textit{configurability}, \textit{adaptability}, \textit{consistency}, and \textit{explainability}.

%Modeling traceability suffers \textbf{peculiar limitations}:
%\begin{itemize}
%    \item Existing separatism in current approaches
%    \begin{itemize}
%        \item Modeling knowledge about traceability is scattered among many (specific) %metamodels
%        \item Existing approaches to traceability are domain, and/or language, and/or %target specific
%    \end{itemize}
%    \item Lack of quality concerns
%    \begin{itemize}
%        \item Existing generic approaches do not consider quality aspects 
%        \item AI buzz (literature reviews) and industrial neglect
%        \item Why is explainability not (yet) further investigated? 
%    \end{itemize}
%\end{itemize}
%Table \ref{table:occurence-of-qualities} shows the occurrence of the above mentioned %traceability qualities in the selection of related approaches found in the literature. As %can be seen..



\begin{table}[h] 
\addtolength{\leftskip} {-0.25cm}
\begin{tabular}{l|c|c|c|c|c}
\textbf{Approaches}\textbackslash{}\textbf{Quality} & 
\textbf{Adaptability} & \textbf{Granularity} & \textbf{Consistency} & \textbf{Confidence} & \textbf{Explainability} \\ \hline 

Goknil \textit{et al.}% 
\cite{goknil2008-metamodel-for-reasoning}    & 
Generic types  &  1-step links & --  &   --  &  --   \\

Taromirad \textit{et al.}% 
\cite{taromirad2012-TIM-for-Agile-req}     &  
Fixed types &  1-step links & -- &  -- & -- \\

Haidrar \textit{et al.}% 
\cite{Haidrar_2018}     &  
Fixed types & 1-step links & Timeliness & -- & -- \\

Sannier \textit{et al.}% 
\cite{sannier2012-TIM-for-text-req-in-MDE}     &  
Specific types  & 1-step links  & --  & -- & -- \\

Dubois \textit{et al.}% 
\cite{dubois2010-TIM-for-Req-in-MD-Automotive}     & 
Specific types & 1-step links & -- & -- & --\\

Sanchez \textit{et al.}% 
\cite{Sanchez_2011}     & 
Specific types & Multi steps & -- & -- & Evidences \\

Yrjonen \textit{et al.}% 
\cite{yrjonen2010-TIM-for-NonFunc-in-MDE}     &  
Specific types & Multi steps & Timeliness & -- & Evidences \\

Jimenez  \textit{et al.}% 
\cite{Jim_nez_2013}     &    
Specific types & Multi steps & (not applicable) & -- & --  \\

Levendovsky \textit{et al.}% 
\cite{levendovsky2010-TIM-for-MT}     &  
Generic types & 1-step links & Context sensitive & -- & Evidences \\

Wang \textit{et al.}% 
\cite{wang2020-TIM-for-NL-to-AADL}     &   
Fixed types & 1-step links & -- & -- & --  \\

Carniel \textit{et al.}% 
\cite{carniel2018-TIM-for-Impact-analysis-agile}     &   
Fixed types & Multi steps & -- & -- & -- \\ \hline

Spanoudakis \textit{et al.}% 
\cite{spanoudakis2004-rule-based-generation-of-req-traceability-relations}     &    
Generic types &                  &            & --  & -- \\

Pavalkis \textit{et al.}% 
\cite{pavalkis2017-TIM-for-BPMN}     &  
Specific types & 1-step links & --  & --  & Agent \\

Maté \textit{et al.}% 
\cite{mate2011-TIM-for-MDA-data-warehouses}     &  
Specific types & 1-step links & -- & --  & -- \\ \hline

Azevedo  \textit{et al.}% 
\cite{azevedo2019-traceability-metamodel-and-reference-model}     &    
Generic types & Multi steps & Timeliness  & --  & --\\

HeiSig \textit{et al.}% 
\cite{heisig2019-generic-traceability-metamodel-end-to-end-capra}     &    
Generic types & Compositional & Context sensitive & --  & -- 
\end{tabular}

\caption{Occurrences of the main properties for modeling traceability.}
\label{table:occurence-of-qualities}
\end{table}

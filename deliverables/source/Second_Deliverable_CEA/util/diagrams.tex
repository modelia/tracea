\usepackage{tikz}
\usetikzlibrary{arrows}
\usetikzlibrary{shapes}
\usetikzlibrary{patterns}
\usetikzlibrary{positioning}
\usetikzlibrary{calc}
\usetikzlibrary{trees}

\newcommand*{\stereotype}[1]{
  <<{#1}>>
}

\newcommand*{\datatype}[1]{
  \stereotype{datatype}\\
  #1
}

\newcommand*{\enum}[1]{
  \stereotype{enumeration}\\
  #1
}

\newcommand*{\obj}[2]{
  \underline{{#1}~:~{#2}}
}

\newcommand*{\attributes}[1]{
  \nodepart{second}
  \begin{tabular}{@{}l@{}}
  #1
  \end{tabular}
}

\newcommand*{\literals}[1]{
  \nodepart{second}
  \ttfamily
  \begin{tabular}{@{}l@{}}
  #1
  \end{tabular}
}

\newcommand*{\operations}[1]{
  \nodepart{third}
  \begin{tabular}{@{}l@{}}
  #1
  \end{tabular}
}

\tikzstyle{reportcolor}=[
  draw=black,
  line width = 0.4pt
]

\tikzstyle{box}=[
  reportcolor,
  rectangle,
  align=center,
  minimum width=5em
]

\tikzstyle{abstractbox}=[box,
  every text node part/.style={
    font=\itshape
  }
]

\tikzstyle{class}=[box,
  rectangle split,
  rectangle split parts=2,
  minimum width=9em,
  rectangle split part align={center, left}
]

\tikzstyle{classop}=[class,
  rectangle split parts=3,
  rectangle split part align={center, left, left}
]
  
\tikzstyle{object}=[class]

\tikzstyle{abstractclass}=[class,
  every text node part/.style={
    font=\itshape
  }
]

\tikzstyle{enum}=[class]

\tikzstyle{datatype}=[class]

\tikzstyle{inherits}=[reportcolor, ->, > = open triangle 90]

\tikzstyle{contains}=[reportcolor, diamond->, > = angle 45]

\tikzstyle{assoc}=[reportcolor, ->, > = angle 45]

\tikzstyle{biassoc}=[reportcolor, <->, > = angle 45]



\subsection{Conclusion}\label{sec:conclusion}

Our survey reveals the continuous interest in traceability even if, often, is considered as a second class citizen. Our analysis also highlights several aspects of current traceability methods that should be further developed, especially if we want to use traceability techniques to improve ongoing AI explainability initiatives. 

To reach these conclusions we have conducted two surveys. The first one focused on all the works devoted to traceability methods and techniques while the second one looked at the increasing number of papers proposing AI techniques to solve software engineering problems to see if those works were integrating (or not) explainability concerns. They do not\footnote{We see the application of AI to automatically identify new traces, e.g.~\cite{clelandhuang2014-traceability-trends-and-futurte-direction} but not the opposite, i.e. the use of traceability to help explain AI components}, which makes this traceability survey even more relevant. 

Moreover, we argue that our feature model helps map the research area and will help understand and standardize traceability approaches and theories. This should help understand, compare and evaluate new traceability proposals. Indeed, there is a prominent call for generalization and standardization~\cite{wohlrab2020-traceability-organization-process-culture} in this field. Gaps between traceability-related research subfields could be better filled in with a common language bringing better communication.

We put a peculiar focus on the distinction between the modeling of traceability and the use of trace links and support our claims with two surveys. The first shows actual trends in traceability literature figuring a prominent call for generalization and standardization~\cite{wohlrab2020-traceability-organization-process-culture}. Current limitations are bond by technical, conceptual, and cultural gaps between research fields that could be fill in with a common language for a better communication. We argue that our feature model helps map the research area and will help understand and standardize traceability approaches and theories. The second survey shows that traceability, recognized as a strong weapon for verification and accountability enhancement, has not got much attention in AI-enabled software engineering. Instead, most work aiming to augment the automation level of traceability offer technical improvement in the automated derivation of links between artefacts of heterogeneous nature - and more specifically text artefacts~\cite{clelandhuang2014-traceability-trends-and-futurte-direction}.
We see this as a urgent call for more work on traceability for the explainability of AI.



\section{Conclusion}\label{sec:conclusion}
% \begin{descriptioncompact}
%     \item[Protocol] We propose a protocol to evaluate traceability solution
%     \item[Application] We show an application on Capra.
%     \item[Extension] We detail the extension of Capra as an example of Tracea integration.
%     \item[Limitations] We show Capra's design limitations
% \end{descriptioncompact}

This deliverable presents means to integrate quality traceability into SysMLv2 using Trace\textit{a}.
With Trace\textit{a} datatypes, SysMLv2 relationships can be annotated with valuable information related to their quality -- \textit{i.e.,} their \textit{trustability}. 
The degree of confidence as well as the information necessary to justify it can be associated to SysML links (connections) through metadata definition.


This integration is \textit{orthogonal}. It does not impact the structure of the language itself -- changes happen at the model level with new feature libraries, not at the metamodel level.
This allows the (re)use of the machinery supporting (meta)annotations and eases the (re)definition of tracing structures specific to a certain project or company. We showcase these benefits in one small example that also reveals the current limitations of SysMLv2's implementation.

The SST confirms that annotating features are valuable and salient artefacts in the development of SysML. They shall take more and more importance in the future releases. Work on the evaluation of expressions at the model level is highly required and will be part of the agenda of the fourth quarter of 2021. 
Finally, as a sign of encouragement, the SST invites us to further investigate in the direction we took. 

\subsection{State of the art} \label{sec:rw}
We present and discuss in this section works related to ours to motivate and justify the need of a new survey in this field. %We will start by exposing the meta-studies dealing with the literature relating to traceability approaches in general. Then, we will present the meta-studies whose purpose is to make the state of the art of approaches using artificial intelligence to assist in the identification of trace links. 

An initial survey of the work done to increase the automation of traceability in software systems was published in 2007 with a great impact~\cite{clelandhuang2007bestPracticeForAutomatedTraceability}. Authors describe nine best practices to augment the potential of traceability that will remain until today the long reach goal to achieve in the emerging research community. They call for a clear establishment of traceability purposes. They also call for an effort from researchers to define the granularity they need to achieve their goals. Researchers must as well keep in mind what will require the integration of this ideal tooling into native software environments. 
On a concrete level, requirement traceability needs quality specifications and exact term definitions to express clearly the conceptualization occurring among them. This requires rich content. Rich enough to bridge the cognitive gaps that plague transitions from one lifecycle phase to the next. Rich enough as well to connect intra-domain semantic synonyms and keep a coherent hierarchy between traced concepts. 
The same year, Galvao \textit{et al.} surveys approaches addressing MDE limitations regarding traceability~\cite{galvao2007-survey-traceability-in-MDE}. In this work, authors show the enormous potential of MDE to represent traces, to generate them automatically, and to assist their management and maintenance. MDE is a suitable paradigm to reach for the end-to-end traceability ideal. Authors describe in great details on the usages of traceability in MDE in a distinctive comparison of the different existing approaches. They conclude that a lack in design and specification modeling, as well as a lack in traceability metamodeling, hampers the automation of traceability techniques through the software life cycle in the MDE paradigm. 

These synthetic works preceded a peak in the number of investigations mentioning explicitly traceability. (In \Fig{fig:publicationyears-tm}, we see that in 2010 22 studies mentioned traceability and modeling.) 
Among them, Winkler \textit{et al.} published an in-depth exploration of both traceability in requirement engineering and model-based development~\cite{winkler2010-survey-traceability-and-MDE}. 
Authors point out that researchers in the two research fields do not communicate enough between each others. This hampers the establishment of common rich and ubiquitous traceability techniques. The authors points out at a lack of tooling in a field where everything is set to support automated traceability. Automated transformations are the very best soil to grow end-to-end traceability at a core level. Yet, even in the modeling community, traceability practices are consider far from mature. There is ongoing research to model the traceability process, but there lacks a pro-active sharing between the larger communities researchers belong to (such as requirement engineering, program understanding, and so on). This piece of work is the closest to ours and we aim at continuing this work with taking account of new trends and describing a more functional approach.

Since studies dedicated to traceability come from specific areas, the general understanding is long to converge. All and every literature synthesis call for more common terminology to allow the sharing of scientific results and help augment the maturity of the emerging research field~\cite{winkler2010-survey-traceability-and-MDE}. On an industrial perspective, some industrial interviewees did not even know what traceability actually is about and rose concerns about its salience during the development process~\cite{vale2017-SPL-traceability-a-SMS,clelandhuang2014-traceability-trends-and-futurte-direction}. Traceability is a major concern in the development of safety critical system, but since it is mandatory it does not gain much enthusiasm. As Mader \textit{et. al.} show in their work, traceability has been considered poorly (by non-traceability stakeholders and researchers) and would benefit a better understanding of its purposes. This would help gather evidences (if they still lack) of the importance of tracing abilities during software development and maintenance~\cite{mader2009-motivation-matters-in-traceability-practitioner-survey}. 

Based on these previous works and the disruptive popularity of AI in software engineering, we believe there is a need for a survey that: 1- summarizes and reflects on the changes and new proposals in the last decade and 2 - goes beyond describing the proposals and makes an effort to synthesize their information in a common traceability glossary, metamodel and feature model to help better characterize the area, application scenarios and potential impact on the growing field of AI-enabled software engineering.

%We try to overcome the aforementioned limitations by proposing a complete study comprising a description of the hierarchy of concepts related to traceability, an articulation of the existing configurations coming from the various engineering domains in which traceability participates, and a holistic characterization of traceability application domains. In this sense, our paper differs from previous research as it offers an actionable view of traceability features and purposes. 





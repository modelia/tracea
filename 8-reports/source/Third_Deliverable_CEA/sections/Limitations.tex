\section{Limitations of the Capra solution}\label{sec:limitations}

\sideboxbegin{o}
This section shows Capra's limitations in term of software design. The singleton syndrome and the absence of link editor being the most salient limitations to plan an industrial exploitation of the tool.
\sideboxend

The ability of Capra to represent traces and customize relationships and artefacts is promising. Nonetheless, the tool implementation suffers important limitations. In this section, we present architecture flaws that would restrain an industrial exploitation of Capra.  We show the technical limitations at design and code level that need to be addressed.

\subsection{Singleton syndrome}
The main limitation of Capra lies in the design choices made to implement traces. Its architecture implies that using Capra remains only a one shot deal. There is one file to store links. "Traces", or better said \textit{the trace}, is regenerated from this file at startup. If the user wants to create different \textit{traces}, s.he will have to start a new instance of Capra with a new XMI file.
This is what we call the \textit{singleton syndrome} where a software (OO) is designed around a unique (often static) instance of its main concept. Capra follows this pattern.

To address this limitation would require a general lift of the manipulation and exploration of traces and links. It would also require modifying high-level architecture of the tool. A report explaining the limitation has been sent to the shareholders of Capra. This issue is under current inspection by the development team. 

\subsection{Single step trace links}
The next limitation is that Capra does not offer any opportunity to synthesized the trace structure. The trace is derived from the trace links on demand for visualization only. 
The way a trace is derived can be configured freely in the trace model implementation (see \verb|org.eclipse.capra.TraceMetaModelAdapter|) but can only describe one trace due to the previous limitation.

A solution could be buffering derived traces to avoid reconstructing them to often. Otherwise, without optimization, the tool will hardly scale.

\subsection{Trace edition shortfall}
We use the XMI Reflexive Model Editor to edit trace links and attribute them confidence value and properties. This shall be improved in the future because for now, there is no assertion that the over views on the trace instance remain consistent between one another. In other wother words, modifying trace links through this editor implies to restart Eclipse to take these changing in consideration. 

The development team is aware of this limitation but does not plan to address it, considering that the edition of trace links is of no use.

\subsection{Naming convention and general quality}
The use of a \texttt{Connection} adapter for links is sometimes misleading. In the PlantUML viewer, the trace links show their corresponding origins and targets whereas EMF elements show their internal structure. The later is defined in the wrappers of the artefact model as described in Section \ref{sec:evaluationcapra}.  

"Trace", "Link", "Connection", "Connections", "TraceLink" sometimes collide in method and attribute signatures.  \textit{E.g.}, In \verb|AbstractTraceMetamodel.createTrace| and  \verb|AbstractTrace-| \verb|Metamodel.deleteTrace| "trace" entities are actually connections. Which is fundamentally misleading.

There can be found cut-n-pastes as well which we should be careful about, specifically in the generic trace model (\verb|org.eclipse.capra.generic.tracemodel|), and the generation of PlantUML diagrams (in \verb|org.eclipse.capra.ui.plantuml.DiagramTextProviderHandler|).


% \subsection{Singleton syndrome} 
% \subsection{Architecture flaws}
% \begin{descriptioncompact}
% \item[Single step trace links] 

% \item[Concerns overlap] 


% \item[Naming convention and general quality] 


% \end{descriptioncompact}



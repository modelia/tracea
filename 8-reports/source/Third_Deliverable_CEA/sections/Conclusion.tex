\section{Conclusion}\label{sec:conclusion}
% \begin{descriptioncompact}
%     \item[Protocol] We propose a protocol to evaluate traceability solution
%     \item[Application] We show an application on Capra.
%     \item[Extension] We detail the extension of Capra as an example of Tracea integration.
%     \item[Limitations] We show Capra's design limitations
% \end{descriptioncompact}

In this document we propose a protocol to evaluate solutions to traceability. Five main dimensions are considered: customizability, identification means, visualization and retrieval features, persistence and edition of traces, and trace quality considerations. Capra responded positively to most of the requirements we defined - excepted the quality dimension. 
We also sketch a protocol to integrate Trace\textit{a} in order to extend solutions to traceability suffering such limitation (related to trace quality) and we apply it to Capra. The overall result is promising, more since Capra is ready for Papyrus development. The tool is ready, with some tuning improvement, for case studies and empirical experiments.
Nonetheless, Capra shows design flaws that would need be addressed to target an industrial environment. If the singleton syndrome is being addressed by the development team, the edition of trace artefact is not on plan and this is a serious impediment to Capra's usability.



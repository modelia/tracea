\section{Conclusion}\label{sec:conclusion}

Our survey reveals a continuous interest in traceability even if, often, it does not have the spotlight it deserves\footnote{As an example, a trace-based paper was awarded the most influential paper in the past 10 years at ICSE ~\cite{ko2008-whyline-debugging}. The work introduced a novel trace-based approach to debugging. Yet, the focus was on the debugging aspect of the paper even if traceability was the key to achieve that debugging improvement. The word "trace" alone is mentioned 46 times in the 10 pages paper.} given the key role it plays in a number of software engineering tasks. Work relating to traceability is indeed disseminated within established research communities (e.g., debugging, SPL). Existing conceptualizations vary greatly depending on the community to which its authors belong to as well as the objectives they aim at. As a consequence, a clear and measurable idea of the costs and benefits to software traceability is slow to emerge

To help visualize, classify and compare the different traceability approaches, we propose a feature model covering all important traceability aspects, as derived from a thorough analysis of the traceability literature. Following the existing body of work, we put special emphasis in separating how traces are represented from how they are identified and managed. 

Beyond the feature model, our analysis highlights several limitations of current traceability approaches that should be further developed. Especially given the new challenges the growing use of AI in Software Engineering \cite{shafiq2020machine,watson2020systematic} is introducing (e.g. in terms of reproduciblity and explainability of the AI decisions). In this sense, we hope this paper serves as a ``wake-up call'' to make sure any new AI proposal comes together with a proper traceability mechanism that assists engineers in recording and understanding the impact of the new AI components in the software engineering process.

As further work, we plan to work on some of the roadmap items above, starting with the proposal of a general traceability metamodel (kind of a superset of all the surveyed ones) that could be used as a starting point in any new traceability project. To facilitate the reuse of such metamodel, we will also release the modeling infrastructure to adapt/refine/deploy it. Once we have this core element, we plan to start working with some of the authors of other proposals to map and bridge their algorithms and techniques to this ``unified'' metamodel and study how to embed it in other modeling languages (like UML or SysML) to further facilitate its adoption.
%Current limitations are bond by technical, conceptual, and semantic gaps between research fields. The gaps between traceability-related research subfields could be better filled in with a common language to bring better communication.
 %~\cite{wohlrab2020-traceability-organization-process-culture} 


%We answer the call for more generalization and standardization of traceability with a feature model.
%We argue that our feature model helps map the research area and will help recognize and standardize traceability approaches and theories. This should help understand, compare and evaluate new traceability proposals. 


\section{Motivation}\label{sec:motivation}

To illustrate better the relevance and challenges of implementing traceability appraoches in practice, \ugh{especially in complex CPS environments}, we motivate our words with concrete industrial examples in safety critical software development. We introduce three experiences from our \nb{First mention}{industrial partner} as concrete examples of limitations of current traceability practice.  


\subsection{Manual preference}
Stakeholders and lead architects are unanimous about the need for a robust, accurate, and agile tracing potential through software life cycle. As a matter of fact, it is obvious to any interlocutor that a tool able to elicit any link between two or more artefact from a software system (documentation, certification, code, configuration, design, test, and so on.) is a must for prevention and maintenance. Yet, this tool has not emerged and tracing activities are performed in an \textit{ad hoc} manner. 
As an example, a quality audit at a supplier of our industrial partner resulted in the manual inspection of a huge amount of artefacts in a system. The massive human workload to extract relevant traces and update loads of spreadsheets to commit to SPICE certification~\cite{SPICE-ISO15504} was preferred to automated approaches available. 


\subsection{Product lifecycle management first}
In sequential development, tracing phases and tasks is explicit. Between each phase, when products are passed from one task to the next figures a check point where constraints can be set to maintain the overall legibility of product, information, and management flows. 
When development becomes collaborative, with concurrent tasks performed in parallel, these check points can no longer survive. Phases and tasks overlap and concurrent versions emerge.
To ensure the quality of the development through the whole life cycle, concurrent collaboration requires tracing abilities between phases and tasks, people, processes, and products~\cite{ghaisas2019-traceability-for-a-knowledge-driven-SW}. 
Yet, when time comes to take a decision on which software is best fit to manage a nuclear plan project, it is clear that a mature tooling with traceability as a core concern is not available. This is unfortunate because it would figure a consequent plus value in this kind of project where safety and security are the most prominent concern. 
To develop internally such a software is not possible either - our partner is not an information technology company, and cannot bare such heavy and complex development. This case involves hundreds actors, dozens industrial partners, and produces hundreds requirements, hundreds software products, and thousands relationship between them.
Alternatively, using a certified software from a trusted vendor makes better return on investment (ROI), requires much less burden, and responsibilities.
Our industrial partner was thus rather inclined to invest in a product lifecycle management (PLM) software, serviced by a leader in software development instead of creating its own traceability-first software product. 

Product lifecycle management tools (PLMs) are meant to ease the edification of transversal link through the different phases of a (colossal) project. 
The emphasis here is put on requirements (safety critical) and gives lead architects access to a broad tracing potential between process phases and trustees. Yet, first, these links must be made by hand, and second, links between processes and products - the actual software running the plant - is "lost" in the process. It is not kept real time up-to-date and a gradual decay plagues traces integrity following the hectic innovation pace.
%Synchronizing systems for synchronizing people for synchronizing productivity between phases and tasks.

\subsection{Machine learning alchemy}
Code is versatile. Changes are made at many different levels - requirements change, feedback from testing procedures emerge, design is refactored. Moreover, artefacts involved during software changes are heterogeneous and monitoring trace links between them is an open research topic. 
The hegemonic interest for AI capabilities in all scientific domain has reach traceability investigations at this crossroad between various types of artefact. Software engineering literature shows that AI offers a strong potential to derive abstract information related to the domain of application as well as to the engineering implementation. As such, it helps link together artefacts written in natural language such as documentation or certifications, with artefacts written in programming languages or represented as design models. In recent year, work on AI for domain contextualization has taken the lead of traceability research\cite{clelandhuang2014-traceability-trends-and-futurte-direction,borg2014-SmS-IR-for-traceability}. 
Yet, if our industrial partners mentions investigations with deep learning techniques to augment traceability potential, these projects are failures that the company orders to internships as research side projects.

\subsubsection{Conclusion}
These examples show that traceability is still an emerging field of research. Lots of work has been done but little effort has been put on synthesizing and standardizing research results and industrial experiences. 
Mader \textit{etal.} coined it bluntly a decade ago: "As a community, we know little more about the traceability practice in companies today than we did a decade ago"~\cite{mader2009-motivation-matters-in-traceability-practitioner-survey}.
Today, traceability is recognized essential to quality software but lack the language to form a coherent research field.





\begin{abstract}
Traceability helps explaining the execution and evolution of software systems and it is a key input in many software engineering tasks such as program understanding, maintenance and debugging.
Several metamodels to facilitate the representation of traces and links between related
artefacts have been proposed. There exists a plethora of approaches that focus on distinct segments of the software development processes and products.
Nevertheless, we claim they lack the mechanisms to express important traceability aspects such as the quality of traces, their gradual decay, and the evidences supporting them. This affects the benefits traceability can bring to the above-mentioned tasks. 
This paper presents a more expressive traceability metamodel, covering all the missing dimensions in a single, but extensible and modular, design. This modularity facilitates the integration of our solution in other modeling languages or its partial adoption when only some specific traceability aspects are needed. Its extensibility facilitates its customization (\textit{e.g.,} in terms of the types of links and artefacts) to better cover specific domains.




%Traceability needs a common language.
%Adaptability means edition/extension of metamodel; thus no common metamodel because burden to adapt MM+tooling (trace representation + wrappers). 
%In any case, so long targeting a common language made researchers forget to pay attention to trace quality and accountability. (Two features still embryonic in SE in general).

%We propose 
%i) to disambiguate genericity and adaptability;
%ii) to put together adaptability and quality in a common language.


	\begin{keywords}
		Software Engineering, Model-Driven Development, Traceability, Metamodeling
	\end{keywords}
\end{abstract}


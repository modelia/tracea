% --------------------------------------------------------------------
% OCL definitions for LaTeX
% --------------------------------------------------------------------

% --------------------------------------------------------------------
% Package used by the OCL definitions
% --------------------------------------------------------------------
%\usepackage{fancybox}

% --------------------------------------------------------------------
% Text styles
% --------------------------------------------------------------------

% Style for OCL constraints
%\newcommand{\OCLtext}{\sf}
\newcommand{\OCLtext}{}{\tt}

% Specific style for OCL keywords
%\newcommand{\OCLkeyword}[1]{{\OCLtext \textbf{#1}}}
\newcommand{\OCLkeyword}[1]{{\OCLtext #1}}
% Specific style for OCL comments
\newcommand{\OCLtextcomment}[1]{ {\OCLtext \emph{#1}} }

% --------------------------------------------------------------------
% Environments for OCL constraints
% --------------------------------------------------------------------

% Box-less OCL environment
\newenvironment{ocl}[1][\linewidth]%
  {\begin{minipage}{#1} \OCLtext }%
  {\end{minipage}}
% Boxed OCL environment  
\newenvironment{ocl-boxed}[1][\linewidth]%
  {\begin{Sbox}\begin{minipage}{#1} \OCLtext }%
  {\end{minipage}\end{Sbox}{\setlength\fboxsep{7pt}\cornersize{0.3}\ovalbox{\TheSbox}}}

% --------------------------------------------------------------------
% OCL language elements
% --------------------------------------------------------------------

% Keywords
% body, init, derive are not keywords according to the spec (?)
\newcommand{\OCLand}{\OCLkeyword{and}}
\newcommand{\OCLattr}{\OCLkeyword{attr}}
\newcommand{\OCLbody}{\OCLkeyword{body}}
\newcommand{\OCLcontext}{\OCLkeyword{\textbf{context}}}
\newcommand{\OCLdef}{\OCLkeyword{def}}
\newcommand{\OCLderive}{\OCLkeyword{derive}}
\newcommand{\OCLelse}{\OCLkeyword{else}}
\newcommand{\OCLendif}{\OCLkeyword{endif}}
\newcommand{\OCLendpackage}{\OCLkeyword{endpackage}}
\newcommand{\OCLif}{\OCLkeyword{if}}
\newcommand{\OCLimplies}{\OCLkeyword{implies}}
\newcommand{\OCLin}{\OCLkeyword{in}}
\newcommand{\OCLinit}{\OCLkeyword{init}}
\newcommand{\OCLinv}{\OCLkeyword{\textbf{inv}}}
\newcommand{\OCLlet}{\OCLkeyword{let}}
\newcommand{\OCLnot}{\OCLkeyword{not}}
\newcommand{\OCLoper}{\OCLkeyword{oper}}
\newcommand{\OCLor}{\OCLkeyword{or}}
\newcommand{\OCLpackage}{\OCLkeyword{package}}
\newcommand{\OCLpost}{\OCLkeyword{post}}
\newcommand{\OCLpre}{\OCLkeyword{pre}}
\newcommand{\OCLself}{\OCLkeyword{self}}
\newcommand{\OCLthen}{\OCLkeyword{then}}
\newcommand{\OCLxor}{\OCLkeyword{xor}}

% ->
%\newcommand{\OCLarrow}{$\rightarrow$}
\newcommand{\OCLarrow}{$-{>}$}


% Comments
\newcommand{\OCLcomment}[1]{{--}{--} \OCLtextcomment{#1}}

% --------------------------------------------------------------------
% Simple OCL constraints
% --------------------------------------------------------------------

% Precondition
\newcommand{\OCLprecond}[2]{\OCLcontext #1 \\ \OCLpre #2}
% Postcondition
\newcommand{\OCLpostcond}[2]{\OCLcontext #1 \\ \OCLpost #2}
% Pre and postcondition
\newcommand{\OCLprepostcond}[3]{\OCLcontext #1 \\ \OCLpre #2 \\ \OCLpost #3}
% Invariant
\newcommand{\OCLinvariant}[2]{\OCLcontext #1\OCLinv \\ #2}
% Indent a paragraph
\newcommand{\OCLindent}[1]{{\parbox{8pt}{~} \hfill \parbox[b]{\linewidth}{\vspace{2pt}#1} }}
% If-then-else
\newcommand{\OCLifthenelse}[3]{\OCLif #1 \OCLthen \\ \OCLindent{#2} \OCLelse \\ \OCLindent{#3}  \OCLendif}
% If-then
\newcommand{\OCLifthen}[2]{\OCLif #1 \OCLthen \\ \OCLindent{#2} \OCLendif}

